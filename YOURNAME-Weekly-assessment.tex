\documentclass[10pt,a4paper]{article} 
\usepackage{fancyhdr}
\usepackage{geometry}
\usepackage{lipsum}
\usepackage{lscape}
%% paper geometry
\textwidth 17 true cm
\textheight 24 true cm
\addtolength{\hoffset}{-1.2cm}
\addtolength{\voffset}{-1cm}


%----------------- fancy headers -------------%
                                                                        
\fancyhf{}
\fancyhead[R]{\sffamily Weekly self-assessment sheet}
%\fancyfoot[R]{\sffamily\small{\thepage}}
\fancyhead[L]{\sffamily\small{Name: YOURNAME}}
%\fancyfoot[L]{\sffamily\small{MSIAM -- Ensimag -- UGA}}
\renewcommand{\headrulewidth}{0.2pt} %0.4
\renewcommand{\footrulewidth}{0.2pt} %0
\addtolength{\headheight}{0.pt}
     
     
\newcommand{\tabwid}{.8cm}

\begin{document} 


\pagestyle{fancy}
 
 
 

% comment this lines to gain place:
(the tabular will extend with your comments, just write in the appropriate column)\\
(remove the example lines when you got the point)\\


\noindent\begin{tabular}{|c|p{.85\textwidth}|}
\hline
Period & Comments, remarks, complaints, excuses, requests, questions...\\
\hline
\hline
example & Lorem ipsum dolor sit amet, consectetur adipiscing elit. Phasellus eu velit ut metus malesuada congue. Sed porta risus eu fringilla ornare. Vivamus a sem convallis, dapibus libero ac, vestibulum mi. Praesent ac feugiat libero, sit amet tempus neque. Suspendisse imperdiet sed orci eget luctus. Proin lectus mauris, auctor sed erat eu, viverra vulputate erat. Aenean sagittis sollicitudin lectus ut ultrices. Quisque elementum elit eget magna consectetur ultricies. Suspendisse tincidunt risus ut molestie vulputate. Mauris mattis sem accumsan ante rutrum, ac congue nisi tincidunt. Suspendisse potenti. Etiam vitae sollicitudin magna, quis vulputate arcu. Praesent id dignissim elit. Fusce pulvinar, felis ut imperdiet placerat, arcu orci pellentesque nibh, id interdum neque quam facilisis diam. Nullam scelerisque aliquam elit.\\
\hline
8/10-14/10 &Difficulties to really understand the expected value and the variance of a continuious function.\\ 
\hline
15/10-21/10 &A good understanding of this chapter, I already have worked on Markov chain and gaussian distribution.\\ 
\hline
22/10-4/11 &I did well the chapter's exercise and I have understand this chapter.\\ 
\hline
5/11-25/11 &Big difficulties with the Kalman filter, I don't really understand how it work.\\
\hline
26/11-22/11 &\\
\hline
3/12-9/12 &\\
\hline
10/12-16/12  &\\
\hline
17/12-6/01&\\
\hline
7/01-13/01 &\\
\hline
14/01-20/01 &\\
\hline
\end{tabular}\\



% comment this line to gain place:
%(please remove the example column when you understand how to fill the sheet)
%\begin{landscape}
\noindent\begin{tabular}{|p{.21\textwidth}|p{\tabwid}|p{\tabwid}|p{\tabwid}|p{\tabwid}|p{\tabwid}|p{\tabwid}|p{\tabwid}|p{\tabwid}|p{\tabwid}|p{\tabwid}|}
\hline
Period, week &  8/10-14/10 & 15/10-21/10 & 22/10-4/11 & 5/11-25/11 & 26/11-22/11 & 3/12-9/12 & 10/12-16/12  & 17/12-6/01 & 7/01-13/01 & 14/01-20/01 \\ 
\hline
\hline
% do not forget then to remove one ``&'' per line in the sequel:
Homework duration for this period in hours (expected: at least the same duration as the course you had during the period)& 2hours &1hour&3hour&1hour&&& &&&\\

\hline
Implication in class (0: absent, 1: inactive, 2: acceptable, 3: fully engaged) & 2  &2&3&2&&&&&& \\

\hline
Group organisation (0: you worked alone, 1: disorganised, 2: acceptable, 3: very effective) & 1 &2&2&1&&&&&& \\

\hline
Course content understanding (0: nothing, 1: very complicated, 2: acceptable, 3: everything is clear)& 2 &3&3&1&&&&&& \\

\hline
Math skills for the exercises (0: difficult to even understand the solution, 1: you did nothing alone but mostly understood the solution, 2: bits done alone, the rest is ok thanks to  the solution, 3: most of the work is done alone and essentially correct, 4: easy, do not need solution) & 3 &3&4&1&&&&&& \\

\hline
Research on the web to further your comprehension of the course and/or advance on your final exam (0: uh? which exam?, 1: a bit of googling, but nothing deep, 2: some time spend looking and reading, 3: extensive work) & 1 &2&2&1&&&&&& \\

\hline
Overall assessment of your engagement in this class (0: none, 1: weak, 2: acceptable, 3: exceeds expectations, 4: outstanding) & 2 &2&3&2&&&&&& \\

\hline
%Overall assessment of your home- and class-work for this period (0: did nothing, 1: poor, 2: acceptable, 3: exceeds expectations, 4: outstanding) & 3 &&&&&&&&&&& \\
Overall assessment of your home- and class-work for this period (0: did nothing, 1: poor, 2: acceptable, 3: exceeds expectations, 4: outstanding) & 2 &2&3&1&&&&&& \\

\hline
\end{tabular}
%\end{landscape}
\end{document}




 
 
 
 
 
 
 
 
 
